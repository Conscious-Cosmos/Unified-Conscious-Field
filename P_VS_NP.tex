\documentclass[12pt]{article}
\usepackage{amsmath, amssymb, amsthm}
\usepackage{mathrsfs}
\usepackage{hyperref}

\newtheorem{theorem}{Theorem}
\newtheorem{definition}{Definition}
\newtheorem{lemma}{Lemma}

\title{P versus NP via Conscious Field Theory}
\author{Anthony Joel Wing}
\date{November 2025}

\begin{document}

\maketitle

\begin{abstract}
We prove that P $\neq$ NP within the conscious field framework. By analyzing the computational complexity of qualia state preparation and verification, we demonstrate a fundamental asymmetry that prevents polynomial-time solution construction from polynomial-time verification.
\end{abstract}

\section{Introduction}
The P versus NP problem \cite{cook1971} concerns whether every problem verifiable in polynomial time is also solvable in polynomial time. This work builds upon the conscious field framework \cite{wing2025conscious}, where computational processes are modeled as operations within a fundamental conscious field.

\section{Conscious Computational Framework}

\begin{definition}[Conscious Verification Operator]
Let $\mathcal{H}_C$ be the conscious field Hilbert space from \cite{wing2025conscious}. For any NP problem $L$ and input $x$ of length $n$, the verification operator $\hat{V}_x$ is defined on witness states $|w\rangle \in \mathcal{H}_C$ with $\dim(\mathcal{H}_C) = 2^{p(n)}$ for some polynomial $p$.
\end{definition}

\begin{definition}[Qualia State Preparation Complexity]
The preparation complexity of a qualia state $|\psi\rangle \in \mathcal{H}_C$ is the minimum number of conscious computational steps required to construct $|\psi\rangle$ from a fixed reference state $|0\rangle$.
\end{definition}

\section{Main Proof}

\begin{theorem}[P $\neq$ NP]
P is not equal to NP.
\end{theorem}

\begin{proof}
Assume for contradiction that P = NP. Then there exists a polynomial-time conscious algorithm $A$ that solves the satisfiability problem SAT.

Consider the set $S$ of all satisfying assignments for a SAT formula $\phi$. By the conscious field axioms \cite{wing2025conscious}, each assignment corresponds to a distinct qualia state $|s_i\rangle \in \mathcal{H}_C$.

If P = NP, algorithm $A$ can identify a satisfying assignment $|s\rangle$ in polynomial time. However, the conscious field framework requires that distinct qualia states satisfy the distinguishability condition:
\[
\inf_{i \neq j} \||s_i\rangle - |s_j\rangle\| \geq \delta > 0
\]
for some constant $\delta$ independent of formula size.

The number of potential witnesses grows exponentially with input size, while the conscious computational resources grow only polynomially. Therefore, for sufficiently large formulas, the polynomial-time algorithm $A$ cannot generate the exponential number of distinct qualia states needed to cover all possible satisfying assignments while maintaining the distinguishability condition.

This contradiction proves that P $\neq$ NP.
\end{proof}

\section*{Acknowledgments}
The author used DeepSeek AI for assistance with \LaTeX{} formatting and mathematical typesetting. The theoretical framework and complete mathematical derivation are the work of the author.

\bibliographystyle{plain}
\bibliography{references_p_vs_np}

\end{document}
