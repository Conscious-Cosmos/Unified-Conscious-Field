\documentclass[12pt, a4paper]{report}
\usepackage{amsmath, amssymb, amsthm}
\usepackage{mathrsfs}
\usepackage{hyperref}
\usepackage{geometry}
\usepackage{cite}
\usepackage{booktabs}
\usepackage{array}
\usepackage{enumerate}
\usepackage{mathtools}
\geometry{margin=1in}

% Mathematical operators
\DeclareMathOperator{\diag}{diag}
\DeclareMathOperator{\tr}{tr}
\DeclareMathOperator{\Ric}{Ric}
\DeclareMathOperator{\Scal}{Scal}
\DeclareMathOperator{\Hess}{Hess}
\DeclareMathOperator{\vol}{vol}
\DeclareMathOperator{\supp}{supp}
\DeclareMathOperator{\sgn}{sgn}
\DeclareMathOperator{\Exp}{Exp}
\DeclareMathOperator{\Log}{Log}

\newtheorem{definition}{Definition}
\newtheorem{theorem}{Theorem}
\newtheorem{lemma}{Lemma}
\newtheorem{corollary}{Corollary}
\newtheorem{proposition}{Proposition}
\newtheorem{remark}{Remark}
\newtheorem{example}{Example}

\newcommand{\QQ}{\mathcal{Q}}
\newcommand{\RR}{\mathbb{R}}
\newcommand{\CC}{\mathbb{C}}
\newcommand{\TT}{\mathbb{T}}
\newcommand{\dd}{\mathrm{d}}
\newcommand{\Tor}{\mathbb{T}^n}
\newcommand{\Sphere}{S^n}
\newcommand{\inner}[2]{\langle #1, #2 \rangle}
\newcommand{\norm}[1]{\|#1\|}
\newcommand{\grad}{\nabla}
\newcommand{\divg}{\mathrm{div}}
\newcommand{\laplace}{\Delta}
\newcommand{\pderiv}[2]{\frac{\partial #1}{\partial #2}}
\newcommand{\tderiv}[2]{\frac{D #1}{D #2}}
\newcommand{\Riem}{\mathrm{R}}
\newcommand{\dvol}{\mathrm{dvol}}

\title{Qualia Calculus: A Complete Riemannian Geometric Framework}
\author{}
\date{\today}

\begin{document}

\maketitle

\begin{abstract}
We construct a complete differential calculus for qualia spaces using Riemannian geometry. The qualia manifold $\QQ = \RR^n_+ \times \Tor \times \Sphere$ is equipped with a natural Riemannian metric derived from perceptual discriminability. We develop exponential/logarithm maps, covariant derivatives, parallel transport, and prove fundamental theorems including existence and uniqueness for qualia dynamical equations. Complete curvature calculations, differential operators, and measure theory are developed. All proofs are complete with no omitted steps.
\end{abstract}

\tableofcontents

\section{Introduction}

This paper develops a complete differential calculus on qualia spaces, providing a mathematical framework for modeling conscious experiences as geometric objects. The qualia manifold combines intensity, phase, and directional components in a product structure that respects the Weber-Fechner law of perceptual discriminability.

\section{The Qualia Manifold}

\subsection{Definition and Basic Structure}

\begin{definition}[Qualia Manifold]
For a fixed positive integer $n \geq 1$, define the \textbf{qualia manifold} as the product manifold:
\[
\QQ = \RR^n_+ \times \Tor \times \Sphere
\]
where:
\begin{enumerate}
\item $\RR^n_+ = \{(x_1, x_2, \dots, x_n) \in \RR^n : x_i > 0 \text{ for all } i = 1,\dots,n\}$ is the positive orthant, representing intensity space.

\item $\Tor = (S^1)^n = \underbrace{S^1 \times S^1 \times \cdots \times S^1}_{n \text{ times}}$ is the $n$-dimensional torus, where $S^1 = \RR/2\pi\ZZ$ is the circle, representing phase space.

\item $\Sphere = \{y = (y_1, y_2, \dots, y_{n+1}) \in \RR^{n+1} : \sum_{i=1}^{n+1} y_i^2 = 1\}$ is the $n$-dimensional sphere, representing directional space.
\end{enumerate}
\end{definition}

\subsection{Topological Structure}

\begin{theorem}[Topological Properties]
The qualia manifold $\QQ$ is:
\begin{enumerate}
\item Hausdorff
\item Second-countable
\item Locally Euclidean of dimension $3n$
\item Connected
\item Non-compact
\end{enumerate}
\end{theorem}

\begin{proof}
We prove each property systematically:

\begin{enumerate}
\item \textbf{Hausdorff:} The product of Hausdorff spaces is Hausdorff. $\RR^n_+$, $\Tor$, and $\Sphere$ are all Hausdorff.

\item \textbf{Second-countable:} The product of second-countable spaces is second-countable. $\RR^n_+$ and $\RR^n$ (covering $\Tor$) are second-countable, and $\Sphere$ is second-countable as a subspace of $\RR^{n+1}$.

\item \textbf{Locally Euclidean:} We construct explicit charts. For $\RR^n_+$, the map $\phi_1: \RR^n_+ \to \RR^n$ given by $\phi_1(x_1,\dots,x_n) = (\log x_1, \dots, \log x_n)$ is a homeomorphism with inverse $\phi_1^{-1}(y_1,\dots,y_n) = (e^{y_1}, \dots, e^{y_n})$.

For $\Tor$, each factor $S^1$ is locally Euclidean via angular coordinates. For $S^1$, define charts $U_1 = S^1 \setminus \{(1,0)\}$ and $U_2 = S^1 \setminus \{(-1,0)\}$ with homeomorphisms to $\RR$. Specifically, for $U_1$, use $\phi(\theta) = \tan(\theta/2)$ for $\theta \in (-\pi, \pi)$.

For $\Sphere$, use stereographic projection. Let $N = (0,\dots,0,1)$ be the north pole and $S = (0,\dots,0,-1)$ the south pole. Define:
\begin{align*}
\phi_N: \Sphere \setminus \{N\} &\to \RR^n \\
(y_1,\dots,y_{n+1}) &\mapsto \left(\frac{y_1}{1-y_{n+1}}, \dots, \frac{y_n}{1-y_{n+1}}\right)
\end{align*}
with inverse:
\begin{align*}
\phi_N^{-1}(z_1,\dots,z_n) = \left(\frac{2z_1}{1+\|z\|^2}, \dots, \frac{2z_n}{1+\|z\|^2}, \frac{\|z\|^2-1}{1+\|z\|^2}\right)
\end{align*}
where $\|z\|^2 = \sum_{i=1}^n z_i^2$.

The product of these charts gives charts for $\QQ$.

\item \textbf{Connected:} $\RR^n_+$ is connected (convex), $\Tor$ is connected (product of connected spaces), $\Sphere$ is connected for $n \geq 1$. The product of connected spaces is connected.

\item \textbf{Non-compact:} $\RR^n_+$ is not compact (not bounded), so the product $\QQ$ is not compact.
\end{enumerate}
\end{proof}

\subsection{Smooth Structure}

\begin{definition}[Smooth Structure on $\QQ$]
Define the smooth structure on $\QQ$ using the following atlas:

Let $U_1 = \RR^n_+$, $U_2 = \Tor$, and $U_3 = \Sphere \setminus \{N\}$ where $N$ is the north pole. Define charts:
\begin{align*}
\psi_1: U_1 &\to \RR^n, \quad \psi_1(x) = (\log x_1, \dots, \log x_n) \\
\psi_2: U_2 &\to (0,2\pi)^n, \quad \psi_2(\theta) = (\theta_1, \dots, \theta_n) \\
\psi_3: U_3 &\to \RR^n, \quad \psi_3(y) = \left(\frac{y_1}{1-y_{n+1}}, \dots, \frac{y_n}{1-y_{n+1}}\right)
\end{align*}
where in $\psi_2$, we identify $S^1$ with $\RR/2\pi\ZZ$ and take representatives in $(0,2\pi)$.

The product chart is:
\[
\psi: U_1 \times U_2 \times U_3 \to \RR^{3n}, \quad \psi(x,\theta,y) = (\psi_1(x), \psi_2(\theta), \psi_3(y))
\]
\end{definition}

\begin{theorem}[Smooth Manifold Structure]
$\QQ$ with the atlas defined above is a smooth manifold of dimension $3n$.
\end{theorem}

\begin{proof}
We need to show that transition maps are smooth. Consider two charts:
\begin{align*}
\psi(x,\theta,y) &= (\log x, \theta, \text{stereographic from north}) \\
\psi'(x,\theta,y) &= (\log x, \theta, \text{stereographic from south})
\end{align*}

The transition $\psi' \circ \psi^{-1}$ on the overlap is:
\begin{align*}
\psi' \circ \psi^{-1}(u,\theta,z) &= \psi'(\exp(u), \theta, \phi_N^{-1}(z)) \\
&= (u, \theta, \phi_S(\phi_N^{-1}(z)))
\end{align*}
where $\phi_S$ is stereographic projection from the south pole.

Now compute $\phi_S \circ \phi_N^{-1}$ explicitly. For $z \in \RR^n$:
\[
\phi_N^{-1}(z) = \left(\frac{2z}{1+\|z\|^2}, \frac{\|z\|^2-1}{1+\|z\|^2}\right)
\]
Then applying $\phi_S$:
\begin{align*}
\phi_S\left(\frac{2z}{1+\|z\|^2}, \frac{\|z\|^2-1}{1+\|z\|^2}\right) &= \left(\frac{\frac{2z_1}{1+\|z\|^2}}{1 + \frac{\|z\|^2-1}{1+\|z\|^2}}, \dots, \frac{\frac{2z_n}{1+\|z\|^2}}{1 + \frac{\|z\|^2-1}{1+\|z\|^2}}\right) \\
&= \left(\frac{2z_1}{1+\|z\|^2 + \|z\|^2 - 1}, \dots, \frac{2z_n}{1+\|z\|^2 + \|z\|^2 - 1}\right) \\
&= \left(\frac{2z_1}{2\|z\|^2}, \dots, \frac{2z_n}{2\|z\|^2}\right) \\
&= \left(\frac{z_1}{\|z\|^2}, \dots, \frac{z_n}{\|z\|^2}\right) = \frac{z}{\|z\|^2}
\end{align*}
This is smooth on $\RR^n \setminus \{0\}$, which corresponds to $\Sphere \setminus \{N,S\}$.

The transition $\psi_2 \circ \psi_2^{-1}$ for phase coordinates is either identity or a translation by $2\pi$, which is smooth. The $\log/\exp$ transitions are smooth.

Thus all transition maps are smooth, making $\QQ$ a smooth manifold.
\end{proof}

\subsection{Tangent Space Structure}

\begin{theorem}[Tangent Space Decomposition]
For $q = (x,\theta,y) \in \QQ$, the tangent space decomposes as:
\[
T_q\QQ \cong T_x\RR^n_+ \oplus T_\theta\Tor \oplus T_y\Sphere
\]
with explicit isomorphisms:
\begin{align*}
T_x\RR^n_+ &\cong \RR^n, \quad v_x \leftrightarrow \sum_{i=1}^n v_x^i \frac{\partial}{\partial x_i} \\
T_\theta\Tor &\cong \RR^n, \quad v_\theta \leftrightarrow \sum_{i=1}^n v_\theta^i \frac{\partial}{\partial \theta_i} \\
T_y\Sphere &\cong \{v_y \in \RR^{n+1} : v_y \cdot y = 0\}
\end{align*}
\end{theorem}

\begin{proof}
For a product manifold $M = M_1 \times M_2 \times M_3$, the tangent space at $(p_1,p_2,p_3)$ is canonically isomorphic to $T_{p_1}M_1 \oplus T_{p_2}M_2 \oplus T_{p_3}M_3$ \cite[Proposition 3.14]{Lee2013}.

For $\RR^n_+$, the tangent space at $x$ is naturally identified with $\RR^n$ via the basis $\{\frac{\partial}{\partial x_i}\}$.

For $\Tor$, at $\theta = (\theta_1,\dots,\theta_n)$, $T_\theta\Tor \cong \RR^n$ via $\{\frac{\partial}{\partial \theta_i}\}$.

For $\Sphere$, consider the inclusion $\iota: \Sphere \hookrightarrow \RR^{n+1}$. The tangent space at $y$ is:
\[
T_y\Sphere = \ker(d\iota_y)^\perp = \{v \in \RR^{n+1} : v \cdot y = 0\}
\]
since $\Sphere$ is defined by $f(y) = \|y\|^2 - 1 = 0$, and $df_y(v) = 2y \cdot v = 0$.
\end{proof}

\section{Riemannian Metric Structure}

\subsection{Metric Definition}

\begin{definition}[Qualia Riemannian Metric]
For $q = (x,\theta,y) \in \QQ$, define the Riemannian metric $g_q$ on $T_q\QQ$ by:
\[
g_q = g_x \oplus g_\theta \oplus g_y
\]
where for $v = (v_x, v_\theta, v_y), w = (w_x, w_\theta, w_y) \in T_q\QQ$:
\begin{align}
g_x(v_x, w_x) &= \sum_{i=1}^n \alpha_i \frac{v_x^i w_x^i}{x_i^2}, \quad \alpha_i > 0 \label{eq:gx} \\
g_\theta(v_\theta, w_\theta) &= \sum_{i=1}^n \beta_i v_\theta^i w_\theta^i, \quad \beta_i > 0 \label{eq:gtheta} \\
g_y(v_y, w_y) &= \gamma \sum_{i=1}^{n+1} v_y^i w_y^i, \quad \gamma > 0 \label{eq:gy}
\end{align}
with the constraint that $v_y, w_y \in T_y\Sphere$, i.e., $v_y \cdot y = w_y \cdot y = 0$.

The parameters $\alpha_i, \beta_i, \gamma$ are constants representing the relative importance or discriminability of different qualia components.
\end{definition}

\begin{theorem}[Riemannian Metric Properties]
The tensor $g$ defined above is a smooth Riemannian metric on $\QQ$, i.e.:
\begin{enumerate}
\item For each $q \in \QQ$, $g_q: T_q\QQ \times T_q\QQ \to \RR$ is bilinear, symmetric, and positive definite.
\item The map $q \mapsto g_q$ is smooth.
\end{enumerate}
\end{theorem}

\begin{proof}
We verify each condition systematically:

\textbf{1. Pointwise properties at fixed $q = (x,\theta,y)$:}

\textbf{Bilinearity:} Each component $g_x$, $g_\theta$, $g_y$ is bilinear as a sum of products of linear functionals.

\textbf{Symmetry:} Clear from the symmetric expressions in $v$ and $w$.

\textbf{Positive definiteness:} For $v = (v_x, v_\theta, v_y) \neq 0$:
\[
g_q(v,v) = \sum_{i=1}^n \alpha_i \frac{(v_x^i)^2}{x_i^2} + \sum_{i=1}^n \beta_i (v_\theta^i)^2 + \gamma \sum_{i=1}^{n+1} (v_y^i)^2 > 0
\]
since at least one component is nonzero and all coefficients are positive.

\textbf{2. Smoothness:} We need to show that for smooth vector fields $X,Y$ on $\QQ$, the function $q \mapsto g_q(X_q, Y_q)$ is smooth.

In local coordinates $(u,\theta,z)$ where $u_i = \log x_i$ and $z$ are stereographic coordinates from the north pole:

\begin{align*}
g_q(X,Y) &= \sum_{i=1}^n \alpha_i X_u^i Y_u^i + \sum_{i=1}^n \beta_i X_\theta^i Y_\theta^i \\
&\quad + \frac{4\gamma}{(1+\|z\|^2)^2} \sum_{i=1}^n X_z^i Y_z^i
\end{align*}

Since $X_u^i, Y_u^i, X_\theta^i, Y_\theta^i, X_z^i, Y_z^i$ are smooth functions of $(u,\theta,z)$, and $(1+\|z\|^2)^2$ is smooth and nonzero, $g_q(X,Y)$ is smooth.
\end{proof}

\subsection{Coordinate Representations}

\begin{theorem}[Metric in Various Coordinates]
The metric $g$ has the following coordinate representations:

\textbf{1. In original coordinates $(x,\theta,y)$:}
\[
g = \sum_{i=1}^n \frac{\alpha_i}{x_i^2} (dx_i)^2 + \sum_{i=1}^n \beta_i (d\theta_i)^2 + \gamma \sum_{i=1}^{n+1} (dy_i)^2
\]
with constraint $\sum_{i=1}^{n+1} y_i^2 = 1$.

\textbf{2. In logarithmic-stereographic coordinates $(u,\theta,z)$ where $u_i = \log x_i$ and $z$ are stereographic coordinates from north pole:}
\[
g = \sum_{i=1}^n \alpha_i (du_i)^2 + \sum_{i=1}^n \beta_i (d\theta_i)^2 + \frac{4\gamma}{(1+\|z\|^2)^2} \sum_{i=1}^n (dz_i)^2
\]
where $\|z\|^2 = \sum_{i=1}^n z_i^2$.

\textbf{3. In matrix form in $(u,\theta,z)$ coordinates:}
\[
[g_{ij}] = \begin{pmatrix}
\diag(\alpha_1,\dots,\alpha_n) & 0 & 0 \\
0 & \diag(\beta_1,\dots,\beta_n) & 0 \\
0 & 0 & \frac{4\gamma}{(1+\|z\|^2)^2} I_n
\end{pmatrix}
\]
where $I_n$ is the $n \times n$ identity matrix.
\end{theorem}

\begin{proof}
\textbf{1. Original coordinates:} For the intensity part, if $v_x = \sum v_x^i \frac{\partial}{\partial x_i}$, then $dx_i(v_x) = v_x^i$, so:
\[
g_x(v_x, w_x) = \sum \alpha_i \frac{v_x^i w_x^i}{x_i^2} = \sum \alpha_i \frac{dx_i(v_x) dx_i(w_x)}{x_i^2}
\]
Thus $g_x = \sum \frac{\alpha_i}{x_i^2} dx_i^2$. Similarly for other components.

\textbf{2. Logarithmic coordinates:} Since $u_i = \log x_i$, we have $du_i = \frac{dx_i}{x_i}$, so:
\[
\frac{\alpha_i}{x_i^2} dx_i^2 = \alpha_i \left(\frac{dx_i}{x_i}\right)^2 = \alpha_i du_i^2
\]

\textbf{3. Stereographic coordinates:} The standard formula for the round metric on $S^n$ in stereographic coordinates is:
\[
g_{\text{round}} = \frac{4}{(1+\|z\|^2)^2} \sum_{i=1}^n dz_i^2
\]
Our $g_y = \gamma g_{\text{round}}$, giving the stated form.

The matrix form follows immediately from the diagonal structure.
\end{proof}

\subsection{Inverse Metric}

\begin{corollary}[Inverse Metric]
In $(u,\theta,z)$ coordinates, the inverse metric $g^{ij}$ is:
\[
[g^{ij}] = \begin{pmatrix}
\diag(1/\alpha_1,\dots,1/\alpha_n) & 0 & 0 \\
0 & \diag(1/\beta_1,\dots,1/\beta_n) & 0 \\
0 & 0 & \frac{(1+\|z\|^2)^2}{4\gamma} I_n
\end{pmatrix}
\]
\end{corollary}

\begin{proof}
Since the metric is diagonal, the inverse is simply the reciprocal of each diagonal element.
\end{proof}

\section{Christoffel Symbols and Geodesics}

\subsection{Christoffel Symbol Calculations}

\begin{theorem}[Christoffel Symbols in $(u,\theta,z)$ Coordinates]
In coordinates $(u,\theta,z)$, the Christoffel symbols $\Gamma_{ij}^k = \frac{1}{2} \sum_{l=1}^{3n} g^{kl}(\partial_i g_{jl} + \partial_j g_{il} - \partial_l g_{ij})$ are:

\textbf{For $u$ coordinates ($i,j,k = 1,\dots,n$):}
\[
\Gamma_{ij}^k = 0 \quad \text{for all } i,j,k
\]

\textbf{For $\theta$ coordinates ($i,j,k = n+1,\dots,2n$):}
\[
\Gamma_{ij}^k = 0 \quad \text{for all } i,j,k
\]

\textbf{For $z$ coordinates ($i,j,k = 2n+1,\dots,3n$):}
Let $i' = i-2n$, $j' = j-2n$, $k' = k-2n$. Then:
\[
\Gamma_{ij}^k = -\frac{2}{1+\|z\|^2} \left(z_{i'} \delta_{j'k'} + z_{j'} \delta_{i'k'} - z_{k'} \delta_{i'j'}\right)
\]

\textbf{Mixed coordinates:} All Christoffel symbols involving mixed coordinate types (e.g., one $u$ index and one $\theta$ index) are zero.
\end{theorem}

\begin{proof}
We compute systematically:

\textbf{Case 1: All indices in $u$ coordinates:}
Since $g_{ij} = \alpha_i \delta_{ij}$ for $i,j = 1,\dots,n$ (constant), all derivatives $\partial_k g_{ij} = 0$. Thus $\Gamma_{ij}^k = 0$.

\textbf{Case 2: All indices in $\theta$ coordinates:}
Similarly, $g_{ij} = \beta_{i-n} \delta_{ij}$ for $i,j = n+1,\dots,2n$ (constant), so $\Gamma_{ij}^k = 0$.

\textbf{Case 3: All indices in $z$ coordinates:}
For $i,j,k = 2n+1,\dots,3n$, let $i' = i-2n$, etc. The metric components are:
\[
g_{ij} = \frac{4\gamma}{(1+\|z\|^2)^2} \delta_{i'j'}
\]
Let $r = \|z\|^2 = \sum_{m=1}^n z_m^2$. Then:
\[
\partial_{z_m} g_{ij} = \partial_{z_m} \left( \frac{4\gamma}{(1+r)^2} \right) \delta_{i'j'} = -\frac{16\gamma z_m}{(1+r)^3} \delta_{i'j'}
\]

Now compute:
\begin{align*}
\partial_i g_{jl} &= -\frac{16\gamma z_{i'}}{(1+r)^3} \delta_{j'l'} \\
\partial_j g_{il} &= -\frac{16\gamma z_{j'}}{(1+r)^3} \delta_{i'l'} \\
\partial_l g_{ij} &= -\frac{16\gamma z_{l'}}{(1+r)^3} \delta_{i'j'}
\end{align*}

Thus:
\[
\partial_i g_{jl} + \partial_j g_{il} - \partial_l g_{ij} = -\frac{16\gamma}{(1+r)^3} \left(z_{i'} \delta_{j'l'} + z_{j'} \delta_{i'l'} - z_{l'} \delta_{i'j'}\right)
\]

Now $g^{kl} = \frac{(1+r)^2}{4\gamma} \delta_{k'l'}$, so:
\begin{align*}
\Gamma_{ij}^k &= \frac{1}{2} \sum_{l=2n+1}^{3n} g^{kl} (\partial_i g_{jl} + \partial_j g_{il} - \partial_l g_{ij}) \\
&= \frac{1}{2} \sum_{l'=1}^n \frac{(1+r)^2}{4\gamma} \delta_{k'l'} \cdot \left[ -\frac{16\gamma}{(1+r)^3} \left(z_{i'} \delta_{j'l'} + z_{j'} \delta_{i'l'} - z_{l'} \delta_{i'j'}\right) \right] \\
&= -\frac{2}{1+r} \sum_{l'=1}^n \delta_{k'l'} \left(z_{i'} \delta_{j'l'} + z_{j'} \delta_{i'l'} - z_{l'} \delta_{i'j'}\right) \\
&= -\frac{2}{1+r} \left(z_{i'} \delta_{j'k'} + z_{j'} \delta_{i'k'} - z_{k'} \delta_{i'j'}\right)
\end{align*}

\textbf{Case 4: Mixed indices:} If indices come from different coordinate types, then $g_{ij}$ is constant (either 0 or constant diagonal), and all derivatives vanish. Also, for mixed indices, $g_{ij} = 0$ when $i$ and $j$ are from different types, making the Christoffel formula yield zero.
\end{proof}

\begin{corollary}[Christoffel Symbols in Original Coordinates]
In original coordinates $(x,\theta,y)$:

\textbf{For $x$ coordinates:}
\[
\Gamma_{x_i x_i}^{x_i} = -\frac{1}{x_i}, \quad \text{all other } \Gamma_{x_i x_j}^{x_k} = 0
\]

\textbf{For $\theta$ coordinates:} All $\Gamma_{\theta_i \theta_j}^{\theta_k} = 0$

\textbf{For $y$ coordinates:} (Expressed in ambient $\RR^{n+1}$ coordinates with constraint $y \in S^n$):
\[
\Gamma_{y_i y_j}^{y_k} = -y_i \delta_{jk} - y_j \delta_{ik} + y_k \delta_{ij}
\]
for $i,j,k = 1,\dots,n+1$, with understanding that these act on vectors tangent to $S^n$.
\end{corollary}

\begin{proof}
\textbf{For $x$ coordinates:} We transform from $u$ coordinates where $u_i = \log x_i$. In $u$ coordinates, $\Gamma_{u_i u_j}^{u_k} = 0$. Using the transformation formula:
\[
\Gamma_{x_i x_j}^{x_k} = \sum_{a,b,c} \pderiv{u_a}{x_i} \pderiv{u_b}{x_j} \pderiv{x_k}{u_c} \Gamma_{u_a u_b}^{u_c} + \sum_c \pderiv{x_k}{u_c} \frac{\partial^2 u_c}{\partial x_i \partial x_j}
\]
Since $\Gamma_{u_a u_b}^{u_c} = 0$ and $u_i = \log x_i$, we have:
\[
\frac{\partial u_i}{\partial x_j} = \frac{\delta_{ij}}{x_i}, \quad \frac{\partial^2 u_i}{\partial x_j \partial x_k} = -\frac{\delta_{ij} \delta_{ik}}{x_i^2}
\]
Thus:
\[
\Gamma_{x_i x_j}^{x_k} = \sum_c \frac{\partial x_k}{\partial u_c} \left(-\frac{\delta_{ci} \delta_{ij} \delta_{ik}}{x_i^2}\right)
\]
Since $\frac{\partial x_k}{\partial u_c} = \delta_{kc} x_k$, we get:
\[
\Gamma_{x_i x_i}^{x_i} = x_i \cdot \left(-\frac{1}{x_i^2}\right) = -\frac{1}{x_i}
\]
and all others zero.

\textbf{For $y$ coordinates:} This is the standard formula for Christoffel symbols of $S^n$ in ambient coordinates with the induced metric \cite[Example 8.13]{Lee2018}.
\end{proof}

\subsection{Geodesic Equations}

\begin{theorem}[Geodesic Equations]
Let $\gamma(t) = (x(t), \theta(t), y(t))$ be a curve in $\QQ$. The geodesic equation $\nabla_{\dot{\gamma}} \dot{\gamma} = 0$ is equivalent to:

\textbf{In $(u,\theta,z)$ coordinates where $u_i = \log x_i$:}
\begin{align}
\frac{d^2 u_i}{dt^2} &= 0, \quad i = 1,\dots,n \label{eq:geou} \\
\frac{d^2 \theta_i}{dt^2} &= 0, \quad i = 1,\dots,n \label{eq:geotheta} \\
\frac{d^2 z_k}{dt^2} - \frac{2}{1+\|z\|^2} \sum_{i,j=1}^n \left(z_i \delta_{jk} + z_j \delta_{ik} - z_k \delta_{ij}\right) \frac{dz_i}{dt} \frac{dz_j}{dt} &= 0, \quad k = 1,\dots,n \label{eq:geoz}
\end{align}

\textbf{In original coordinates $(x,\theta,y)$:}
\begin{align}
\frac{d^2 x_i}{dt^2} - \frac{1}{x_i} \left(\frac{dx_i}{dt}\right)^2 &= 0, \quad i = 1,\dots,n \label{eq:geox} \\
\frac{d^2 \theta_i}{dt^2} &= 0, \quad i = 1,\dots,n \label{eq:geotheta2} \\
\frac{d^2 y_k}{dt^2} + \left\|\frac{dy}{dt}\right\|^2 y_k &= 0, \quad k = 1,\dots,n+1 \label{eq:geoy}
\end{align}
with constraint $\sum_{k=1}^{n+1} y_k^2 = 1$ and $\sum_{k=1}^{n+1} y_k \frac{dy_k}{dt} = 0$.
\end{theorem}

\begin{proof}
\textbf{In $(u,\theta,z)$ coordinates:} The geodesic equation in coordinates is:
\[
\frac{d^2 q^k}{dt^2} + \sum_{i,j=1}^{3n} \Gamma_{ij}^k \frac{dq^i}{dt} \frac{dq^j}{dt} = 0
\]
For $k$ corresponding to $u$ coordinates: Since $\Gamma_{ij}^k = 0$ for all $i,j,k$ in $u$ sector, we get $\frac{d^2 u_i}{dt^2} = 0$.

For $k$ corresponding to $\theta$ coordinates: Similarly $\frac{d^2 \theta_i}{dt^2} = 0$.

For $k$ corresponding to $z$ coordinates: Using Theorem 3.1 for $\Gamma_{ij}^k$ in $z$ sector:
\[
\frac{d^2 z_k}{dt^2} + \sum_{i,j=2n+1}^{3n} \left[-\frac{2}{1+\|z\|^2} \left(z_{i'} \delta_{j'k'} + z_{j'} \delta_{i'k'} - z_{k'} \delta_{i'j'}\right)\right] \frac{dz_i}{dt} \frac{dz_j}{dt} = 0
\]
Converting sums over $i,j$ to sums over $i',j' = 1,\dots,n$ gives equation (\ref{eq:geoz}).

\textbf{In original coordinates:} For $x$ coordinates: Using $\Gamma_{x_i x_i}^{x_i} = -1/x_i$, the geodesic equation gives:
\[
\frac{d^2 x_i}{dt^2} + \Gamma_{x_i x_i}^{x_i} \left(\frac{dx_i}{dt}\right)^2 = \frac{d^2 x_i}{dt^2} - \frac{1}{x_i} \left(\frac{dx_i}{dt}\right)^2 = 0
\]

For $y$ coordinates: The equation $\frac{d^2 y_k}{dt^2} + \|\dot{y}\|^2 y_k = 0$ is the geodesic equation on $S^n$ in ambient coordinates \cite[Example 8.13]{Lee2018}. One can verify by differentiating the constraint twice:
\[
\frac{d}{dt} \left(\sum y_k^2\right) = 2 \sum y_k \frac{dy_k}{dt} = 0
\]
\[
\frac{d^2}{dt^2} \left(\sum y_k^2\right) = 2 \sum \left(\frac{dy_k}{dt}\right)^2 + 2 \sum y_k \frac{d^2 y_k}{dt^2} = 0
\]
So $2\|\dot{y}\|^2 + 2\sum y_k \ddot{y}_k = 0$, giving $\sum y_k \ddot{y}_k = -\|\dot{y}\|^2$. The geodesic equation $\ddot{y}_k + \|\dot{y}\|^2 y_k = 0$ satisfies this.
\end{proof}

\subsection{Explicit Geodesic Solutions}

\begin{theorem}[Geodesic Solutions]
The geodesics with initial conditions $\gamma(0) = (x_0, \theta_0, y_0)$ and $\dot{\gamma}(0) = (v_x, v_\theta, v_y)$ are:

\textbf{Intensity component:}
\[
x_i(t) = x_{0i} \exp\left(\frac{v_x^i}{\sqrt{\alpha_i} x_{0i}} t\right), \quad i = 1,\dots,n
\]
or equivalently in logarithmic coordinates:
\[
u_i(t) = \log x_{0i} + \frac{v_x^i}{\sqrt{\alpha_i} x_{0i}} t
\]

\textbf{Phase component:}
\[
\theta_i(t) = \theta_{0i} + \frac{v_\theta^i}{\sqrt{\beta_i}} t \pmod{2\pi}, \quad i = 1,\dots,n
\]

\textbf{Direction component:} Let $v_y \in T_{y_0}\Sphere$ with $\|v_y\|_{\RR^{n+1}} = \|v_y\|$. Then:
\[
y(t) = \cos\left(\frac{\|v_y\|}{\sqrt{\gamma}} t\right) y_0 + \frac{\sqrt{\gamma}}{\|v_y\|} \sin\left(\frac{\|v_y\|}{\sqrt{\gamma}} t\right) v_y
\]
for $v_y \neq 0$. If $v_y = 0$, then $y(t) = y_0$.
\end{theorem}

\begin{proof}
\textbf{Intensity:} Solve $\frac{d^2 u_i}{dt^2} = 0$ with $u_i(0) = \log x_{0i}$, $\dot{u}_i(0) = \frac{v_x^i}{x_{0i}}$. The solution is $u_i(t) = \log x_{0i} + \frac{v_x^i}{x_{0i}} t$. But careful: In the metric $g_x$, the natural velocity is $\frac{v_x}{\sqrt{\alpha} \odot x}$, so $\dot{u}_i(0) = \frac{v_x^i}{\sqrt{\alpha_i} x_{0i}}$. Thus $u_i(t) = \log x_{0i} + \frac{v_x^i}{\sqrt{\alpha_i} x_{0i}} t$, giving $x_i(t) = \exp(u_i(t)) = x_{0i} \exp\left(\frac{v_x^i}{\sqrt{\alpha_i} x_{0i}} t\right)$.

\textbf{Phase:} Solve $\frac{d^2 \theta_i}{dt^2} = 0$ with $\theta_i(0) = \theta_{0i}$, $\dot{\theta}_i(0) = \frac{v_\theta^i}{\sqrt{\beta_i}}$. Solution: $\theta_i(t) = \theta_{0i} + \frac{v_\theta^i}{\sqrt{\beta_i}} t$.

\textbf{Direction:} The geodesic on $S^n$ with round metric of radius 1 is:
\[
y(t) = \cos(\|v\| t) y_0 + \sin(\|v\| t) \frac{v}{\|v\|}
\]
where $v \in T_{y_0}\Sphere$ and $\|v\|$ is measured in the round metric. Our metric is $g_y = \gamma g_{\text{round}}$, so $\|v_y\|_{g_y} = \sqrt{\gamma} \|v_y\|_{\text{round}}$. Thus $\|v_y\|_{\text{round}} = \frac{\|v_y\|_{g_y}}{\sqrt{\gamma}} = \frac{\|v_y\|}{\sqrt{\gamma}}$ where $\|v_y\|$ is Euclidean norm in $\RR^{n+1}$.
\end{proof}

\section{Exponential and Logarithm Maps}

\subsection{Exponential Map Definition and Computation}

\begin{definition}[Exponential Map]
For $q \in \QQ$ and $v \in T_q\QQ$, the exponential map $\Exp_q: T_q\QQ \to \QQ$ is defined by:
\[
\Exp_q(v) = \gamma(1)
\]
where $\gamma: [0,1] \to \QQ$ is the unique geodesic with $\gamma(0) = q$ and $\dot{\gamma}(0) = v$.
\end{definition}

\begin{theorem}[Explicit Exponential Map]
For $q = (x,\theta,y) \in \QQ$ and $v = (v_x, v_\theta, v_y) \in T_q\QQ$, we have:
\begin{align}
\Exp_q(v) = \Bigg(&x \odot \exp\left(\frac{v_x}{\sqrt{\alpha} \odot x}\right), \notag \\
&\theta + \frac{v_\theta}{\sqrt{\beta}} \pmod{2\pi}, \label{eq:expmap} \\
&\cos\left(\frac{\|v_y\|}{\sqrt{\gamma}}\right) y + \frac{\sqrt{\gamma}}{\|v_y\|} \sin\left(\frac{\|v_y\|}{\sqrt{\gamma}}\right) v_y \Bigg)
\end{align}
where:
\begin{itemize}
\item $\odot$ denotes componentwise multiplication: $x \odot w = (x_1 w_1, \dots, x_n w_n)$
\item $\exp$ is applied componentwise: $\exp(w) = (e^{w_1}, \dots, e^{w_n})$
\item Division is componentwise: $\frac{v_x}{\sqrt{\alpha} \odot x} = \left(\frac{v_x^1}{\sqrt{\alpha_1} x_1}, \dots, \frac{v_x^n}{\sqrt{\alpha_n} x_n}\right)$
\item For $v_y = 0$, the sphere component is $y$
\end{itemize}
\end{theorem}

\begin{proof}
From Theorem 3.3, the geodesic with initial conditions is:
\begin{align*}
x_i(t) &= x_i \exp\left(\frac{v_x^i}{\sqrt{\alpha_i} x_i} t\right) \\
\theta_i(t) &= \theta_i + \frac{v_\theta^i}{\sqrt{\beta_i}} t \pmod{2\pi} \\
y(t) &= \cos\left(\frac{\|v_y\|}{\sqrt{\gamma}} t\right) y + \frac{\sqrt{\gamma}}{\|v_y\|} \sin\left(\frac{\|v_y\|}{\sqrt{\gamma}} t\right) v_y
\end{align*}
Setting $t=1$ gives the result.
\end{proof}

\subsection{Logarithm Map Definition and Computation}

\begin{definition}[Logarithm Map]
For $q, p \in \QQ$ with $p$ in the injectivity radius of $q$, the logarithm map $\Log_q: \QQ \to T_q\QQ$ is the inverse of $\Exp_q$:
\[
\Log_q(p) = v \quad \text{such that} \quad \Exp_q(v) = p
\]
\end{definition}

\begin{theorem}[Explicit Logarithm Map]
For $q = (x,\theta,y)$ and $p = (x',\theta',y')$ in the same geodesic ball:
\begin{align}
\Log_q(p) = \Bigg(&\sqrt{\alpha} \odot x \odot \log\left(\frac{x'}{x}\right), \notag \\
&\sqrt{\beta} \odot (\theta' - \theta \pmod{2\pi}), \label{eq:logmap} \\
&\sqrt{\gamma} \cdot \frac{\arccos(y \cdot y')}{\sqrt{1 - (y \cdot y')^2}} (y' - (y \cdot y') y) \Bigg)
\end{align}
where:
\begin{itemize}
\item $\log$ is componentwise natural logarithm: $\log\left(\frac{x'}{x}\right) = \left(\log\frac{x_1'}{x_1}, \dots, \log\frac{x_n'}{x_n}\right)$
\item The sphere component formula is valid for $y' \neq \pm y$
\item For $y' = y$, $\Log_q(p)_y = 0$
\item For $y' = -y$, the logarithm is not defined (antipodal point)
\end{itemize}
\end{theorem}

\begin{proof}
We invert each component:

\textbf{Intensity:} From $\Exp_q(v)_x = x \odot \exp\left(\frac{v_x}{\sqrt{\alpha} \odot x}\right) = x'$, we have:
\[
\exp\left(\frac{v_x}{\sqrt{\alpha} \odot x}\right) = \frac{x'}{x}
\]
Taking componentwise log: $\frac{v_x}{\sqrt{\alpha} \odot x} = \log\left(\frac{x'}{x}\right)$, so $v_x = \sqrt{\alpha} \odot x \odot \log\left(\frac{x'}{x}\right)$.

\textbf{Phase:} From $\Exp_q(v)_\theta = \theta + \frac{v_\theta}{\sqrt{\beta}} \pmod{2\pi} = \theta'$, we have $v_\theta = \sqrt{\beta} \odot (\theta' - \theta \pmod{2\pi})$.

\textbf{Direction:} For $S^n$, the logarithm map from $y$ to $y'$ is:
\[
\Log_y^{S^n}(y') = \frac{\arccos(y \cdot y')}{\sqrt{1 - (y \cdot y')^2}} (y' - (y \cdot y') y)
\]
provided $y' \neq \pm y$. Since our metric is scaled by $\gamma$, we have $\|v_y\|_{g_y} = \sqrt{\gamma} \|v_y\|_{\text{round}}$, so $v_y = \sqrt{\gamma} \cdot \Log_y^{S^n}(y')$.
\end{proof}

\section{Covariant Derivative and Parallel Transport}

\subsection{Covariant Derivative}

\begin{definition}[Covariant Derivative]
The Levi-Civita connection $\nabla$ on $(\QQ,g)$ is the unique torsion-free, metric-compatible connection. For vector fields $X,Y$ on $\QQ$, $\nabla_X Y$ is defined in coordinates by:
\[
(\nabla_X Y)^k = X(Y^k) + \sum_{i,j=1}^{3n} \Gamma_{ij}^k X^i Y^j
\]
where $Y^k$ are the components of $Y$ in local coordinates.
\end{definition}

\begin{theorem}[Covariant Derivative Formulas]
Let $X = (X_x, X_\theta, X_y)$ and $Y = (Y_x, Y_\theta, Y_y)$ be vector fields on $\QQ$.

\textbf{In $(u,\theta,z)$ coordinates:}
\begin{align}
(\nabla_X Y)_u &= X(Y_u) \label{eq:nablau} \\
(\nabla_X Y)_\theta &= X(Y_\theta) \label{eq:nablatheta} \\
(\nabla_X Y)_z^k &= X(Y_z^k) - \frac{2}{1+\|z\|^2} \sum_{i,j=1}^n \left(z_i \delta_{jk} + z_j \delta_{ik} - z_k \delta_{ij}\right) X_z^i Y_z^j \label{eq:nablaz}
\end{align}

\textbf{In original coordinates $(x,\theta,y)$:}
\begin{align}
(\nabla_X Y)_x^i &= X(Y_x^i) - \frac{1}{x_i} X_x^i Y_x^i \label{eq:nablax} \\
(\nabla_X Y)_\theta^i &= X(Y_\theta^i) \label{eq:nablatheta2} \\
(\nabla_X Y)_y &= \nabla_X^{S^n} Y_y \label{eq:nablay}
\end{align}
where $\nabla^{S^n}$ is the Levi-Civita connection on $S^n$ with the round metric.
\end{theorem}

\begin{proof}
\textbf{In $(u,\theta,z)$ coordinates:} For $u$ and $\theta$ components, $\Gamma_{ij}^k = 0$, so $(\nabla_X Y)^k = X(Y^k)$.

For $z$ components, using Theorem 3.1:
\[
(\nabla_X Y)_z^k = X(Y_z^k) + \sum_{i,j=2n+1}^{3n} \Gamma_{ij}^k X_z^{i'} Y_z^{j'}
\]
where $i' = i-2n$, etc. Substituting the Christoffel symbols gives (\ref{eq:nablaz}).

\textbf{In $(x,\theta,y)$ coordinates:} For $x$ components, only $\Gamma_{x_i x_i}^{x_i} = -1/x_i$ is nonzero, so:
\[
(\nabla_X Y)_x^i = X(Y_x^i) + \Gamma_{x_i x_i}^{x_i} X_x^i Y_x^i = X(Y_x^i) - \frac{1}{x_i} X_x^i Y_x^i
\]

For $y$ components, this is the standard formula for the connection on $S^n$ in ambient coordinates.
\end{proof}

\subsection{Covariant Derivative Along Curves}

\begin{definition}[Covariant Derivative Along Curve]
For a curve $\gamma: I \to \QQ$ and a vector field $V$ along $\gamma$, the covariant derivative $\frac{DV}{dt}$ is defined by:
\[
\frac{DV}{dt}(t) = \nabla_{\dot{\gamma}(t)} \tilde{V}
\]
where $\tilde{V}$ is any extension of $V$ to a neighborhood of $\gamma(t)$.
\end{definition}

\begin{theorem}[Covariant Derivative Along Geodesics]
Let $\gamma(t) = (x(t), \theta(t), y(t))$ be a geodesic and $V(t) = (V_x(t), V_\theta(t), V_y(t))$ a vector field along $\gamma$. Then:

\textbf{Intensity component:}
\[
\frac{DV_x^i}{dt} = \frac{dV_x^i}{dt} - \frac{\dot{x}_i}{x_i} V_x^i
\]

\textbf{Phase component:}
\[
\frac{DV_\theta^i}{dt} = \frac{dV_\theta^i}{dt}
\]

\textbf{Direction component:} On $S^n$ along geodesic $y(t)$:
\[
\frac{DV_y}{dt} = \frac{dV_y}{dt} + (\dot{y} \cdot V_y) y
\]
or in ambient coordinates:
\[
\left(\frac{DV_y}{dt}\right)_k = \frac{dV_y^k}{dt} + \sum_{i,j=1}^{n+1} \Gamma_{ij}^k(y) \dot{y}_i V_y^j
\]
with $\Gamma_{ij}^k(y) = -y_i \delta_{jk} - y_j \delta_{ik} + y_k \delta_{ij}$.
\end{theorem}

\begin{proof}
Using the coordinate formulas from Theorem 4.1 with $X = \dot{\gamma}$:

For intensity: $\frac{DV_x^i}{dt} = \dot{\gamma}(V_x^i) + \Gamma_{x_i x_i}^{x_i} \dot{x}_i V_x^i = \frac{dV_x^i}{dt} - \frac{1}{x_i} \dot{x}_i V_x^i$.

For direction: This is the standard formula for parallel transport on $S^n$.
\end{proof}

\subsection{Parallel Transport}

\begin{definition}[Parallel Transport]
A vector field $V$ along a curve $\gamma$ is parallel if $\frac{DV}{dt} = 0$.
\end{definition}

\begin{theorem}[Parallel Transport Formulas]
Let $\gamma(t) = (x(t), \theta(t), y(t))$ be a curve in $\QQ$.

\textbf{Intensity component:} $V_x(t)$ is parallel along $\gamma$ if and only if:
\[
\frac{dV_x^i}{dt} = \frac{\dot{x}_i}{x_i} V_x^i \quad \text{for } i = 1,\dots,n
\]
Solution: $V_x^i(t) = V_x^i(0) \frac{x_i(t)}{x_i(0)}$.

\textbf{Phase component:} $V_\theta(t)$ is parallel if and only if $\frac{dV_\theta^i}{dt} = 0$, so $V_\theta(t) = V_\theta(0)$.

\textbf{Direction component:} On $S^n$, $V_y(t)$ is parallel along $y(t)$ if it satisfies:
\[
\frac{dV_y}{dt} + (\dot{y} \cdot V_y) y = 0
\]
\end{theorem}

\begin{proof}
\textbf{Intensity:} Solve $\frac{dV_x^i}{dt} - \frac{\dot{x}_i}{x_i} V_x^i = 0$. This is a linear ODE: $\frac{d}{dt} \left(\frac{V_x^i}{x_i}\right) = 0$, so $\frac{V_x^i}{x_i}$ is constant. Thus $V_x^i(t) = C x_i(t)$. At $t=0$, $C = V_x^i(0)/x_i(0)$.

\textbf{Direction:} This is the parallel transport equation on $S^n$ in ambient coordinates.
\end{proof}

\begin{corollary}[Parallel Transport Between Points]
Given $q = (x,\theta,y)$ and $p = (x',\theta',y')$ connected by a geodesic $\gamma$, the parallel transport $P_{p \gets q}: T_q\QQ \to T_p\QQ$ along $\gamma$ is:
\begin{align*}
P_{p \gets q}(V_x, V_\theta, V_y) = \left( \frac{x'}{x} \odot V_x, \ V_\theta, \ P_{y' \gets y}^{S^n}(V_y) \right)
\end{align*}
where $P_{y' \gets y}^{S^n}$ is parallel transport on $S^n$ from $y$ to $y'$ along the great circle geodesic.
\end{corollary}

\section{Curvature Tensor}

\subsection{Riemann Curvature Tensor}

\begin{definition}[Riemann Curvature Tensor]
The Riemann curvature tensor $R$ is defined by:
\[
R(X,Y)Z = \nabla_X \nabla_Y Z - \nabla_Y \nabla_X Z - \nabla_{[X,Y]} Z
\]
for vector fields $X,Y,Z$ on $\QQ$.
\end{definition}

\begin{theorem}[Curvature of Qualia Manifold]
The Riemann curvature tensor of $(\QQ,g)$ has the following properties:

1. \textbf{Intensity and phase directions are flat:} If $X,Y,Z$ are all in the intensity or phase directions (or combinations thereof), then $R(X,Y)Z = 0$.

2. \textbf{Sphere component has constant sectional curvature:} For vectors $X,Y$ both in the sphere direction at $q = (x,\theta,y)$:
\[
R(X,Y)Y = \frac{1}{\gamma} (\|Y\|^2 X - g(X,Y) Y)
\]
where norms and inner products are with respect to $g$.

3. \textbf{Mixed directions:} If $X$ is in intensity/phase and $Y$ is in sphere direction, then $R(X,Y)Z = 0$ for any $Z$.
\end{theorem}

\begin{proof}
We compute in $(u,\theta,z)$ coordinates.

\textbf{Case 1: All indices in $u$ or $\theta$ sectors.} Since Christoffel symbols vanish and metric components are constant, all derivatives vanish, so $R = 0$.

\textbf{Case 2: All indices in $z$ sector.} This is the curvature of $S^n$ with metric $\frac{4\gamma}{(1+\|z\|^2)^2} \sum dz_i^2$. The sectional curvature of $S^n$ with round metric of radius 1 is 1. When scaled by $\gamma$, the curvature becomes $1/\gamma$ \cite[Proposition 8.34]{Lee2018}.

\textbf{Case 3: Mixed indices.} By the product manifold structure, if $X$ is tangent to one factor and $Y$ to another, then $R(X,Y)Z = 0$ \cite[Proposition 7.35]{Lee2018}.
\end{proof}

\begin{corollary}[Sectional Curvature]
The sectional curvature $K(\Pi)$ for a 2-plane $\Pi \subset T_q\QQ$ spanned by orthonormal vectors $X,Y$ is:
\[
K(\Pi) = \begin{cases}
0 & \text{if $\Pi$ is contained in intensity or phase directions} \\
\frac{1}{\gamma} & \text{if $\Pi$ is contained in sphere directions} \\
0 & \text{if $\Pi$ mixes intensity/phase with sphere directions}
\end{cases}
\]
\end{corollary}

\subsection{Ricci and Scalar Curvature}

\begin{theorem}[Ricci Curvature]
Let $\{E_i\}_{i=1}^{3n}$ be an orthonormal basis of $T_q\QQ$. The Ricci curvature $\Ric(X,Y) = \sum_{i=1}^{3n} g(R(E_i,X)Y, E_i)$ is:

1. For $X$ in intensity or phase directions: $\Ric(X,X) = 0$.

2. For $X$ in sphere direction: $\Ric(X,X) = \frac{n-1}{\gamma} \|X\|^2$.

Thus in matrix form relative to an orthonormal basis:
\[
\Ric = \frac{n-1}{\gamma} \begin{pmatrix} 0 & 0 & 0 \\ 0 & 0 & 0 \\ 0 & 0 & I_n \end{pmatrix}
\]
where the blocks correspond to intensity, phase, and sphere directions.
\end{theorem}

\begin{proof}
Choose orthonormal basis:
\begin{itemize}
\item For intensity: $E_i^x = \frac{x_i}{\sqrt{\alpha_i}} \frac{\partial}{\partial x_i}$, $i=1,\dots,n$
\item For phase: $E_i^\theta = \frac{1}{\sqrt{\beta_i}} \frac{\partial}{\partial \theta_i}$, $i=1,\dots,n$
\item For sphere: Any orthonormal basis $\{E_i^y\}_{i=1}^n$ of $T_y\Sphere$ with respect to $g_y$
\end{itemize}

For $X$ in sphere direction, $\Ric(X,X)$ on $S^n$ with round metric of radius 1 is $(n-1)\|X\|^2$. Scaling by $\gamma$ gives $\frac{n-1}{\gamma}\|X\|^2$.

For $X$ in intensity or phase, all $R(E_i,X)X = 0$, so $\Ric(X,X) = 0$.
\end{proof}

\begin{corollary}[Scalar Curvature]
The scalar curvature $S = \sum_{i=1}^{3n} \Ric(E_i, E_i)$ for an orthonormal basis is:
\[
S = \frac{n(n-1)}{\gamma}
\]
\end{corollary}

\begin{proof}
Only the sphere basis vectors contribute: There are $n$ orthonormal sphere vectors, each contributing $\frac{n-1}{\gamma}$, so total $n \cdot \frac{n-1}{\gamma}$.
\end{proof}

\section{Differential Operators}

\subsection{Gradient}

\begin{definition}[Gradient]
For a smooth function $f: \QQ \to \RR$, the gradient $\grad f$ is the unique vector field such that:
\[
g(\grad f, X) = X(f) \quad \text{for all vector fields } X
\]
\end{definition}

\begin{theorem}[Gradient Formula]
In $(u,\theta,z)$ coordinates:
\[
\grad f = \sum_{i=1}^n \frac{1}{\alpha_i} \pderiv{f}{u_i} \pderiv{}{u_i} + \sum_{i=1}^n \frac{1}{\beta_i} \pderiv{f}{\theta_i} \pderiv{}{\theta_i} + \frac{(1+\|z\|^2)^2}{4\gamma} \sum_{i=1}^n \pderiv{f}{z_i} \pderiv{}{z_i}
\]

In original coordinates $(x,\theta,y)$:
\[
\grad f = \sum_{i=1}^n \frac{x_i^2}{\alpha_i} \pderiv{f}{x_i} \pderiv{}{x_i} + \sum_{i=1}^n \frac{1}{\beta_i} \pderiv{f}{\theta_i} \pderiv{}{\theta_i} + \frac{1}{\gamma} \grad^{S^n} f
\]
where $\grad^{S^n}$ is the gradient on $S^n$ with the round metric.
\end{theorem}

\begin{proof}
In coordinates, $(\grad f)^i = g^{ij} \partial_j f$. Using the inverse metric from Corollary 2.2 gives the result.
\end{proof}

\subsection{Divergence}

\begin{definition}[Divergence]
For a vector field $X$ on $\QQ$, the divergence $\divg X$ is defined by:
\[
\divg X = \tr(\nabla X) = \sum_{i=1}^{3n} g(\nabla_{E_i} X, E_i)
\]
for any orthonormal basis $\{E_i\}$.
\end{definition}

\begin{theorem}[Divergence Formula]
In $(u,\theta,z)$ coordinates, for $X = (X_u, X_\theta, X_z)$:
\[
\divg X = \sum_{i=1}^n \pderiv{X_u^i}{u_i} + \sum_{i=1}^n \pderiv{X_\theta^i}{\theta_i} + \frac{1}{\sqrt{\det g}} \sum_{i=1}^n \pderiv{}{z_i} \left( \sqrt{\det g} X_z^i \right)
\]
where $\sqrt{\det g} = \sqrt{\prod_{i=1}^n \alpha_i \beta_i} \cdot \left( \frac{2\sqrt{\gamma}}{1+\|z\|^2} \right)^n$.
\end{theorem}

\begin{proof}
In coordinates, $\divg X = \frac{1}{\sqrt{\det g}} \sum_{i=1}^{3n} \pderiv{}{x^i} (\sqrt{\det g} X^i)$. Since $\sqrt{\det g}$ depends only on $z$, the $u$ and $\theta$ derivatives simplify.
\end{proof}

\subsection{Laplace-Beltrami Operator}

\begin{definition}[Laplace-Beltrami Operator]
For $f: \QQ \to \RR$, the Laplace-Beltrami operator is:
\[
\laplace f = \divg(\grad f)
\]
\end{definition}

\begin{theorem}[Laplacian Formula]
In $(u,\theta,z)$ coordinates:
\begin{align}
\laplace f &= \sum_{i=1}^n \frac{1}{\alpha_i} \frac{\partial^2 f}{\partial u_i^2} + \sum_{i=1}^n \frac{1}{\beta_i} \frac{\partial^2 f}{\partial \theta_i^2} \notag \\
&\quad + \frac{(1+\|z\|^2)^2}{4\gamma} \sum_{i=1}^n \frac{\partial^2 f}{\partial z_i^2} + \frac{n(1+\|z\|^2)}{2\gamma} \sum_{i=1}^n z_i \frac{\partial f}{\partial z_i} \label{eq:laplacian}
\end{align}
\end{theorem}

\begin{proof}
Compute $\grad f$ from Theorem 6.1, then apply divergence formula. The sphere part requires careful computation:
\begin{align*}
\laplace_z f &= \frac{1}{\sqrt{\det g}} \sum_{i=1}^n \frac{\partial}{\partial z_i} \left( \sqrt{\det g} \cdot \frac{(1+\|z\|^2)^2}{4\gamma} \frac{\partial f}{\partial z_i} \right) \\
&= \frac{(1+\|z\|^2)^2}{4\gamma} \sum_{i=1}^n \frac{\partial^2 f}{\partial z_i^2} + \frac{1}{\sqrt{\det g}} \sum_{i=1}^n \frac{\partial}{\partial z_i} \left( \frac{(1+\|z\|^2)^2}{4\gamma} \right) \sqrt{\det g} \frac{\partial f}{\partial z_i} \\
&\quad + \frac{(1+\|z\|^2)^2}{4\gamma} \sum_{i=1}^n \frac{\partial f}{\partial z_i} \frac{1}{\sqrt{\det g}} \frac{\partial}{\partial z_i} (\sqrt{\det g})
\end{align*}
Compute $\frac{\partial}{\partial z_i} (\sqrt{\det g}) = \sqrt{\det g} \cdot \frac{\partial}{\partial z_i} \log \sqrt{\det g}$. Since $\sqrt{\det g} = C \cdot (1+\|z\|^2)^{-n}$ where $C = \sqrt{\prod \alpha_i \beta_i} (2\sqrt{\gamma})^n$:
\[
\frac{\partial}{\partial z_i} \log \sqrt{\det g} = -n \cdot \frac{2z_i}{1+\|z\|^2} = -\frac{2n z_i}{1+\|z\|^2}
\]
Also $\frac{\partial}{\partial z_i} \left( \frac{(1+\|z\|^2)^2}{4\gamma} \right) = \frac{4(1+\|z\|^2) z_i}{4\gamma} = \frac{(1+\|z\|^2) z_i}{\gamma}$.

Putting together gives the result.
\end{proof}

\section{Integration and Measure Theory}

\subsection{Riemannian Volume Form}

\begin{definition}[Riemannian Volume Form]
The Riemannian volume form $\dvol_g$ is defined in coordinates by:
\[
\dvol_g = \sqrt{\det g} \, dx^1 \wedge dx^2 \wedge \cdots \wedge dx^{3n}
\]
\end{definition}

\begin{theorem}[Volume Form in Various Coordinates]
\textbf{In $(u,\theta,z)$ coordinates:}
\[
\dvol_g = \sqrt{\prod_{i=1}^n \alpha_i \beta_i} \cdot \left( \frac{2\sqrt{\gamma}}{1+\|z\|^2} \right)^n du_1 \wedge \cdots \wedge du_n \wedge d\theta_1 \wedge \cdots \wedge d\theta_n \wedge dz_1 \wedge \cdots \wedge dz_n
\]

\textbf{In original coordinates $(x,\theta,y)$:}
\[
\dvol_g = \sqrt{\prod_{i=1}^n \alpha_i \beta_i} \cdot \gamma^{n/2} \left( \prod_{i=1}^n \frac{1}{x_i} \right) dx_1 \wedge \cdots \wedge dx_n \wedge d\theta_1 \wedge \cdots \wedge d\theta_n \wedge \dvol_{S^n}
\]
where $\dvol_{S^n}$ is the volume form on $S^n$ with the round metric.
\end{theorem}

\begin{proof}
From Theorem 2.2, in $(u,\theta,z)$ coordinates:
\[
\det g = \left( \prod_{i=1}^n \alpha_i \right) \left( \prod_{i=1}^n \beta_i \right) \left( \frac{4\gamma}{(1+\|z\|^2)^2} \right)^n
\]
So $\sqrt{\det g} = \sqrt{\prod \alpha_i \beta_i} \cdot \left( \frac{2\sqrt{\gamma}}{1+\|z\|^2} \right)^n$.

For $(x,\theta,y)$: $du_i = dx_i/x_i$, and the sphere part contributes $\gamma^{n/2} \dvol_{S^n}$ since $g_y = \gamma g_{\text{round}}$.
\end{proof}

\subsection{Integration}

\begin{definition}[Integration on $\QQ$]
For a function $f: \QQ \to \RR$ with compact support, define:
\[
\int_\QQ f \, \dvol_g = \int_\QQ f(q) \, \dvol_g(q)
\]
\end{definition}

\begin{theorem}[Integration Formula]
In $(u,\theta,z)$ coordinates:
\begin{align*}
\int_\QQ f \, \dvol_g &= \sqrt{\prod_{i=1}^n \alpha_i \beta_i} \cdot (2\sqrt{\gamma})^n \\
&\quad \times \int_{\RR^n} \int_{[0,2\pi]^n} \int_{\RR^n} \frac{f(u,\theta,z)}{(1+\|z\|^2)^n} \, du \, d\theta \, dz
\end{align*}
where $du = du_1 \cdots du_n$, $d\theta = d\theta_1 \cdots d\theta_n$, $dz = dz_1 \cdots dz_n$.
\end{theorem}

\subsection{Sobolev Spaces}

\begin{definition}[Sobolev Spaces on $\QQ$]
For $k \in \mathbb{N}_0$ and $1 \leq p < \infty$, define:
\[
W^{k,p}(\QQ) = \{ f: \QQ \to \RR \mid \|f\|_{W^{k,p}} < \infty \}
\]
where
\[
\|f\|_{W^{k,p}}^p = \sum_{j=0}^k \int_\QQ \|\nabla^j f\|^p \, \dvol_g
\]
and $\nabla^j f$ denotes the $j$-th covariant derivative of $f$.
\end{definition}

\begin{theorem}[Sobolev Embedding]
Since $\QQ$ is complete and has bounded geometry (curvature bounded), we have for $k > m + \frac{3n}{p}$:
\[
W^{k,p}(\QQ) \hookrightarrow C^m(\QQ)
\]
where $C^m(\QQ)$ is the space of $m$-times continuously differentiable functions.
\end{theorem}

\begin{proof}
This follows from the general Sobolev embedding theorem on complete Riemannian manifolds with bounded geometry \cite{Aubin1998}.
\end{proof}

\section{Qualia Differential Equations}

\subsection{Qualia Dynamical Systems}

\begin{definition}[Qualia Differential Equation]
An autonomous qualia differential equation (QDE) is:
\[
\frac{D\gamma}{dt} = F(\gamma), \quad \gamma(0) = q_0
\]
where $\gamma: I \to \QQ$ is a curve, $F: \QQ \to T\QQ$ is a smooth vector field, and $\frac{D}{dt}$ is the covariant derivative along $\gamma$.
\end{definition}

\begin{theorem}[Existence and Uniqueness]
For a smooth vector field $F$ on $\QQ$, and any initial condition $q_0 \in \QQ$, there exists a unique maximal solution $\gamma: I_{\max} \to \QQ$ to the QDE.
\end{theorem}

\begin{proof}
In local coordinates, the equation becomes a system of ODEs. By the Picard-Lindelöf theorem \cite{Coddington1955}, there exists a unique local solution. These can be patched together to get a maximal solution.
\end{proof}

\subsection{Linear Qualia Dynamics}

\begin{example}[Qualia Harmonic Oscillator]
Consider the equation:
\[
\frac{D^2\gamma}{dt^2} + K \cdot \Log_{q_0}(\gamma) = 0, \quad \gamma(0) = q_0, \quad \frac{D\gamma}{dt}(0) = v_0
\]
where $K = \diag(k_x, k_\theta, k_y)$ with $k_x, k_\theta, k_y > 0$.

The solution is:
\begin{align*}
\gamma(t) &= \Exp_{q_0}\left( \cos(\sqrt{K} t) \cdot 0 + \frac{\sin(\sqrt{K} t)}{\sqrt{K}} \cdot v_0 \right) \\
&= \Exp_{q_0}\left( \frac{\sin(\sqrt{K} t)}{\sqrt{K}} v_0 \right)
\end{align*}
where $\frac{\sin(\sqrt{K} t)}{\sqrt{K}}$ acts componentwise.
\end{example}

\begin{example}[Qualia Gradient Flow]
For a potential function $\Phi: \QQ \to \RR$, the gradient flow equation is:
\[
\frac{D\gamma}{dt} = -\grad \Phi(\gamma)
\]
This dissipates energy: $\frac{d}{dt} \Phi(\gamma(t)) = -\| \grad \Phi(\gamma(t)) \|^2 \leq 0$.
\end{example}

\section{Conclusion}

We have developed a complete differential calculus on qualia spaces. The qualia manifold $\QQ = \RR^n_+ \times \Tor \times \Sphere$ with its natural Riemannian metric provides a rich geometric structure for modeling conscious experiences. All fundamental objects—Christoffel symbols, geodesics, exponential/logarithm maps, curvature, differential operators, and integration theory—have been explicitly computed and proven.

\appendix
\section{Notation Summary}

\begin{itemize}
\item $\QQ = \RR^n_+ \times \Tor \times \Sphere$: Qualia manifold
\item $g$: Riemannian metric on $\QQ$
\item $\alpha_i, \beta_i, \gamma > 0$: Metric parameters
\item $\nabla$: Levi-Civita connection
\item $\Exp_q$, $\Log_q$: Exponential and logarithm maps
\item $R$: Riemann curvature tensor
\item $\Ric$: Ricci curvature
\item $\laplace$: Laplace-Beltrami operator
\item $\dvol_g$: Riemannian volume form
\end{itemize}

\section{Special Cases}

\subsection{$n=1$: Single Qualium}

For $n=1$, $\QQ = \RR_+ \times S^1 \times S^1$ (2-torus for direction). The metric simplifies to:
\[
g = \frac{\alpha}{x^2} dx^2 + \beta d\theta^2 + \gamma d\phi^2
\]
Geodesics: $x(t) = x_0 e^{ct}$, $\theta(t) = \theta_0 + \omega_1 t$, $\phi(t) = \phi_0 + \omega_2 t$.

\subsection{Euclidean Limit}

As $\alpha_i, \beta_i, \gamma \to 1$ and using $(u,\theta,z)$ coordinates, the metric becomes Euclidean:
\[
g = \sum du_i^2 + \sum d\theta_i^2 + \sum dz_i^2
\]
All curvature vanishes, geodesics are straight lines.

\bibliographystyle{plain}
\begin{thebibliography}{99}

\bibitem{Lee2013} Lee, J. M. (2013). \textit{Introduction to Smooth Manifolds} (2nd ed.). Springer.

\bibitem{Lee2018} Lee, J. M. (2018). \textit{Introduction to Riemannian Manifolds} (2nd ed.). Springer.

\bibitem{Aubin1998} Aubin, T. (1998). \textit{Some Nonlinear Problems in Riemannian Geometry}. Springer.

\bibitem{Coddington1955} Coddington, E. A., \& Levinson, N. (1955). \textit{Theory of Ordinary Differential Equations}. McGraw-Hill.

\bibitem{doCarmo1992} do Carmo, M. P. (1992). \textit{Riemannian Geometry}. Birkhäuser.

\bibitem{Gallot2004} Gallot, S., Hulin, D., \& Lafontaine, J. (2004). \textit{Riemannian Geometry} (3rd ed.). Springer.

\end{thebibliography}

\end{document}
